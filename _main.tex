% Options for packages loaded elsewhere
\PassOptionsToPackage{unicode}{hyperref}
\PassOptionsToPackage{hyphens}{url}
%
\documentclass[
]{book}
\usepackage{amsmath,amssymb}
\usepackage{lmodern}
\usepackage{ifxetex,ifluatex}
\ifnum 0\ifxetex 1\fi\ifluatex 1\fi=0 % if pdftex
  \usepackage[T1]{fontenc}
  \usepackage[utf8]{inputenc}
  \usepackage{textcomp} % provide euro and other symbols
\else % if luatex or xetex
  \usepackage{unicode-math}
  \defaultfontfeatures{Scale=MatchLowercase}
  \defaultfontfeatures[\rmfamily]{Ligatures=TeX,Scale=1}
\fi
% Use upquote if available, for straight quotes in verbatim environments
\IfFileExists{upquote.sty}{\usepackage{upquote}}{}
\IfFileExists{microtype.sty}{% use microtype if available
  \usepackage[]{microtype}
  \UseMicrotypeSet[protrusion]{basicmath} % disable protrusion for tt fonts
}{}
\makeatletter
\@ifundefined{KOMAClassName}{% if non-KOMA class
  \IfFileExists{parskip.sty}{%
    \usepackage{parskip}
  }{% else
    \setlength{\parindent}{0pt}
    \setlength{\parskip}{6pt plus 2pt minus 1pt}}
}{% if KOMA class
  \KOMAoptions{parskip=half}}
\makeatother
\usepackage{xcolor}
\IfFileExists{xurl.sty}{\usepackage{xurl}}{} % add URL line breaks if available
\IfFileExists{bookmark.sty}{\usepackage{bookmark}}{\usepackage{hyperref}}
\hypersetup{
  pdftitle={BSOL R Guide},
  pdfauthor={Archie Atack / Alex Terry},
  hidelinks,
  pdfcreator={LaTeX via pandoc}}
\urlstyle{same} % disable monospaced font for URLs
\usepackage{color}
\usepackage{fancyvrb}
\newcommand{\VerbBar}{|}
\newcommand{\VERB}{\Verb[commandchars=\\\{\}]}
\DefineVerbatimEnvironment{Highlighting}{Verbatim}{commandchars=\\\{\}}
% Add ',fontsize=\small' for more characters per line
\usepackage{framed}
\definecolor{shadecolor}{RGB}{248,248,248}
\newenvironment{Shaded}{\begin{snugshade}}{\end{snugshade}}
\newcommand{\AlertTok}[1]{\textcolor[rgb]{0.94,0.16,0.16}{#1}}
\newcommand{\AnnotationTok}[1]{\textcolor[rgb]{0.56,0.35,0.01}{\textbf{\textit{#1}}}}
\newcommand{\AttributeTok}[1]{\textcolor[rgb]{0.77,0.63,0.00}{#1}}
\newcommand{\BaseNTok}[1]{\textcolor[rgb]{0.00,0.00,0.81}{#1}}
\newcommand{\BuiltInTok}[1]{#1}
\newcommand{\CharTok}[1]{\textcolor[rgb]{0.31,0.60,0.02}{#1}}
\newcommand{\CommentTok}[1]{\textcolor[rgb]{0.56,0.35,0.01}{\textit{#1}}}
\newcommand{\CommentVarTok}[1]{\textcolor[rgb]{0.56,0.35,0.01}{\textbf{\textit{#1}}}}
\newcommand{\ConstantTok}[1]{\textcolor[rgb]{0.00,0.00,0.00}{#1}}
\newcommand{\ControlFlowTok}[1]{\textcolor[rgb]{0.13,0.29,0.53}{\textbf{#1}}}
\newcommand{\DataTypeTok}[1]{\textcolor[rgb]{0.13,0.29,0.53}{#1}}
\newcommand{\DecValTok}[1]{\textcolor[rgb]{0.00,0.00,0.81}{#1}}
\newcommand{\DocumentationTok}[1]{\textcolor[rgb]{0.56,0.35,0.01}{\textbf{\textit{#1}}}}
\newcommand{\ErrorTok}[1]{\textcolor[rgb]{0.64,0.00,0.00}{\textbf{#1}}}
\newcommand{\ExtensionTok}[1]{#1}
\newcommand{\FloatTok}[1]{\textcolor[rgb]{0.00,0.00,0.81}{#1}}
\newcommand{\FunctionTok}[1]{\textcolor[rgb]{0.00,0.00,0.00}{#1}}
\newcommand{\ImportTok}[1]{#1}
\newcommand{\InformationTok}[1]{\textcolor[rgb]{0.56,0.35,0.01}{\textbf{\textit{#1}}}}
\newcommand{\KeywordTok}[1]{\textcolor[rgb]{0.13,0.29,0.53}{\textbf{#1}}}
\newcommand{\NormalTok}[1]{#1}
\newcommand{\OperatorTok}[1]{\textcolor[rgb]{0.81,0.36,0.00}{\textbf{#1}}}
\newcommand{\OtherTok}[1]{\textcolor[rgb]{0.56,0.35,0.01}{#1}}
\newcommand{\PreprocessorTok}[1]{\textcolor[rgb]{0.56,0.35,0.01}{\textit{#1}}}
\newcommand{\RegionMarkerTok}[1]{#1}
\newcommand{\SpecialCharTok}[1]{\textcolor[rgb]{0.00,0.00,0.00}{#1}}
\newcommand{\SpecialStringTok}[1]{\textcolor[rgb]{0.31,0.60,0.02}{#1}}
\newcommand{\StringTok}[1]{\textcolor[rgb]{0.31,0.60,0.02}{#1}}
\newcommand{\VariableTok}[1]{\textcolor[rgb]{0.00,0.00,0.00}{#1}}
\newcommand{\VerbatimStringTok}[1]{\textcolor[rgb]{0.31,0.60,0.02}{#1}}
\newcommand{\WarningTok}[1]{\textcolor[rgb]{0.56,0.35,0.01}{\textbf{\textit{#1}}}}
\usepackage{longtable,booktabs,array}
\usepackage{calc} % for calculating minipage widths
% Correct order of tables after \paragraph or \subparagraph
\usepackage{etoolbox}
\makeatletter
\patchcmd\longtable{\par}{\if@noskipsec\mbox{}\fi\par}{}{}
\makeatother
% Allow footnotes in longtable head/foot
\IfFileExists{footnotehyper.sty}{\usepackage{footnotehyper}}{\usepackage{footnote}}
\makesavenoteenv{longtable}
\usepackage{graphicx}
\makeatletter
\def\maxwidth{\ifdim\Gin@nat@width>\linewidth\linewidth\else\Gin@nat@width\fi}
\def\maxheight{\ifdim\Gin@nat@height>\textheight\textheight\else\Gin@nat@height\fi}
\makeatother
% Scale images if necessary, so that they will not overflow the page
% margins by default, and it is still possible to overwrite the defaults
% using explicit options in \includegraphics[width, height, ...]{}
\setkeys{Gin}{width=\maxwidth,height=\maxheight,keepaspectratio}
% Set default figure placement to htbp
\makeatletter
\def\fps@figure{htbp}
\makeatother
\setlength{\emergencystretch}{3em} % prevent overfull lines
\providecommand{\tightlist}{%
  \setlength{\itemsep}{0pt}\setlength{\parskip}{0pt}}
\setcounter{secnumdepth}{5}
\usepackage{booktabs}
\ifluatex
  \usepackage{selnolig}  % disable illegal ligatures
\fi
\usepackage[]{natbib}
\bibliographystyle{plainnat}

\title{BSOL R Guide}
\author{Archie Atack / Alex Terry}
\date{2021-11-25}

\begin{document}
\maketitle

{
\setcounter{tocdepth}{1}
\tableofcontents
}
\hypertarget{about}{%
\chapter{About}\label{about}}

Chapters on R packages and example code.

To do:
- Leaflet
- Dplyr
-- SQL functions
- Advanced dplyr
-- Pivot functions
-- Across
-- Row number
- SQL / dplyr equivalents
- Stringr
- Loops
- ggplot2
- Email from R

Appendix
- Shortcuts
- Links

\hypertarget{loading-data-from-file}{%
\chapter{Loading data from file}\label{loading-data-from-file}}

\hypertarget{read-from-csv}{%
\section{Read from CSV}\label{read-from-csv}}

\begin{Shaded}
\begin{Highlighting}[]
\FunctionTok{library}\NormalTok{(tidyverse)}
\NormalTok{csv\_data }\OtherTok{\textless{}{-}} \FunctionTok{read\_csv}\NormalTok{(}\StringTok{"data.csv"}\NormalTok{)}
\end{Highlighting}
\end{Shaded}

read\_csv() is found in the readr package (part of tidyverse) and is an improved version of the base R function read.csv().

\begin{Shaded}
\begin{Highlighting}[]
\NormalTok{csv\_data }\OtherTok{\textless{}{-}} \FunctionTok{read\_csv}\NormalTok{(}
  \StringTok{"data.csv"}\NormalTok{,}
  \AttributeTok{col\_types =} \FunctionTok{cols}\NormalTok{( }\CommentTok{\# specify data types}
    \AttributeTok{col1 =} \FunctionTok{col\_character}\NormalTok{(),}
    \AttributeTok{col2 =} \FunctionTok{col\_double}\NormalTok{(),}
    \AttributeTok{col3 =} \FunctionTok{col\_date}\NormalTok{(),}
    \AttributeTok{col4 =} \FunctionTok{col\_datetime}\NormalTok{(),}
    \AttributeTok{col5 =} \FunctionTok{col\_time}\NormalTok{(),}
    \AttributeTok{col6 =} \FunctionTok{col\_logical}\NormalTok{()}
\NormalTok{  )}
\NormalTok{)}
\end{Highlighting}
\end{Shaded}

Use the col\_types argument to specify data types. See the documentation for cols() to see the possible types.

If the col\_types argument is left blank, read\_csv estimates an appropriate data type for each column using the first 1000 rows of data. Amend the guess\_max argument to adjust the number of rows used to estimate type.

Use col\_types = cols() to suppress the output message to the console.

Use cols\_only instead of cols() to only specify the data types of a subset of the columns.

\hypertarget{read-from-excel}{%
\section{Read from excel}\label{read-from-excel}}

\begin{Shaded}
\begin{Highlighting}[]
\FunctionTok{library}\NormalTok{(readxl)}
\NormalTok{excel\_data }\OtherTok{\textless{}{-}} \FunctionTok{read\_xlsx}\NormalTok{(}\StringTok{"data.xlsx"}\NormalTok{)}
\end{Highlighting}
\end{Shaded}

readxl is downloaded as part of the tidyverse packages but needs to be called specifically to load in its functions.

By default, readxl will load in the first sheet of the workbook.

To read in xls files, use either read\_excel() or read\_xls().

\begin{Shaded}
\begin{Highlighting}[]
\NormalTok{excel\_data }\OtherTok{\textless{}{-}} \FunctionTok{read\_xlsx}\NormalTok{(}
  \StringTok{"data.xlsx"}\NormalTok{,}
  \AttributeTok{sheet =} \StringTok{"Sheet1"}\NormalTok{,}
  \AttributeTok{range =} \StringTok{"A1:D20"}
\NormalTok{)}

\NormalTok{excel\_data }\OtherTok{\textless{}{-}} \FunctionTok{read\_xlsx}\NormalTok{(}
  \StringTok{"data.xlsx"}\NormalTok{,}
  \AttributeTok{sheet =} \StringTok{"Sheet1"}\NormalTok{,}
  \AttributeTok{skip =} \DecValTok{3}\NormalTok{,}
  \AttributeTok{n\_max =} \DecValTok{100}
\NormalTok{)}
\end{Highlighting}
\end{Shaded}

Use the sheet, range, skip, n\_max etc arguments to specify the range of data to read.

\begin{Shaded}
\begin{Highlighting}[]
\NormalTok{excel\_data }\OtherTok{\textless{}{-}} \FunctionTok{read\_xlsx}\NormalTok{(}
  \StringTok{"data.xlsx"}\NormalTok{,}
  \AttributeTok{sheet =} \StringTok{"Sheet1"}\NormalTok{,}
  \AttributeTok{range =} \StringTok{"A1:D20"}\NormalTok{,}
  \AttributeTok{col\_types =} \FunctionTok{c}\NormalTok{(}\StringTok{"text"}\NormalTok{,}\StringTok{"numeric"}\NormalTok{,}\StringTok{"date"}\NormalTok{,}\StringTok{"guess"}\NormalTok{)}
\NormalTok{)}
\end{Highlighting}
\end{Shaded}

Use the col\_types argument to specify data types. See the documentation for list of all possible types.

\hypertarget{writing-data-to-file}{%
\chapter{Writing data to file}\label{writing-data-to-file}}

\hypertarget{write-to-csv}{%
\section{Write to CSV}\label{write-to-csv}}

\begin{Shaded}
\begin{Highlighting}[]
\FunctionTok{library}\NormalTok{(tidyverse)}
\FunctionTok{write\_csv}\NormalTok{(data, }\StringTok{"data.csv"}\NormalTok{)}
\end{Highlighting}
\end{Shaded}

write\_csv() is found in the readr package (part of tidyverse) and is an improved version of the base R function write.csv().

Use the na argument to specify how NULL or NA values should be represented.

\hypertarget{write-to-excel}{%
\section{Write to excel}\label{write-to-excel}}

\begin{Shaded}
\begin{Highlighting}[]
\FunctionTok{library}\NormalTok{(openxlsx)}
\FunctionTok{write.xlsx}\NormalTok{(data, }\StringTok{"data.xlsx"}\NormalTok{, }\AttributeTok{sheetName =} \StringTok{"Summary"}\NormalTok{)}
\end{Highlighting}
\end{Shaded}

The openxlsx enables data to be written to excel files.

Use the asTable argument to specify if the data should be stored in a table within excel.

See the write.xlsx documentation for various options to customise the workbook aesthetics from R.

\hypertarget{interact-with-sql-server}{%
\chapter{Interact with SQL Server}\label{interact-with-sql-server}}

\hypertarget{sql-connection}{%
\section{SQL connection}\label{sql-connection}}

\begin{Shaded}
\begin{Highlighting}[]
\FunctionTok{library}\NormalTok{(odbc)}
\NormalTok{sql\_connection }\OtherTok{\textless{}{-}}
  \FunctionTok{dbConnect}\NormalTok{(}
    \FunctionTok{odbc}\NormalTok{(),}
    \AttributeTok{Driver =} \StringTok{"SQL Server"}\NormalTok{,}
    \AttributeTok{Server =} \StringTok{"MLCSU{-}BI{-}SQL"}\NormalTok{,}
    \AttributeTok{Database =} \StringTok{"EAT\_Reporting\_BSOL"}\NormalTok{,}
    \AttributeTok{Trusted\_Connection =} \StringTok{"True"}
\NormalTok{  )}
\end{Highlighting}
\end{Shaded}

odbc is a package that allows R to connect to the SQL server databases.

\hypertarget{read-data-using-select}{%
\section{Read data using select}\label{read-data-using-select}}

\begin{Shaded}
\begin{Highlighting}[]
\FunctionTok{library}\NormalTok{(DBI)}
\FunctionTok{library}\NormalTok{(tidyverse)}
\NormalTok{sql\_data }\OtherTok{\textless{}{-}} 
  \FunctionTok{dbGetQuery}\NormalTok{(}
\NormalTok{    sql\_connection,}
    \StringTok{"SELECT *}
\StringTok{    FROM table"}
\NormalTok{  ) }\SpecialCharTok{\%\textgreater{}\%} \FunctionTok{as\_tibble}\NormalTok{() }\CommentTok{\# converts output to tibble for practicality}
\end{Highlighting}
\end{Shaded}

The DBI package allows R to interact with the data in SQL server.

The dbGetQuery() function outputs the data as a dataframe. Therefore the dataframe is then converted to a tibble using as\_tibble() for practicality (tibbles print tidier outputs to the console).

\hypertarget{read-data-by-running-sql-script}{%
\section{Read data by running SQL script}\label{read-data-by-running-sql-script}}

\begin{Shaded}
\begin{Highlighting}[]
\NormalTok{sql\_data }\OtherTok{\textless{}{-}} 
  \FunctionTok{dbGetQuery}\NormalTok{(}
\NormalTok{    sql\_connection,}
\NormalTok{    readr}\SpecialCharTok{::}\FunctionTok{read\_file}\NormalTok{(}\StringTok{"script.sql"}\NormalTok{)}
\NormalTok{  ) }\SpecialCharTok{\%\textgreater{}\%} \FunctionTok{as\_tibble}\NormalTok{()}
\end{Highlighting}
\end{Shaded}

The read\_file() function from readr allows an existing SQL script in the file directory to be run from RStudio and will load the results of the query into R.

\hypertarget{run-sql-from-r---not-reading-data}{%
\section{Run SQL from R - not reading data}\label{run-sql-from-r---not-reading-data}}

\begin{Shaded}
\begin{Highlighting}[]
\FunctionTok{dbExecute}\NormalTok{(}
\NormalTok{  sql\_connection,}
  \StringTok{"SELECT *}
\StringTok{  INTO \#\#temp}
\StringTok{  FROM table"}
\NormalTok{)}
\end{Highlighting}
\end{Shaded}

Use this when creating temp tables in SQL server or running stored procedures from R.

\hypertarget{create-static-sql-table-from-r}{%
\section{Create static SQL table from R}\label{create-static-sql-table-from-r}}

\begin{Shaded}
\begin{Highlighting}[]
\FunctionTok{dbWriteTable}\NormalTok{(}
\NormalTok{  sql\_connection,}
  \FunctionTok{Id}\NormalTok{(}\AttributeTok{schema =} \StringTok{"dbo"}\NormalTok{, }\AttributeTok{table =} \StringTok{"BSOL\_XXXX\_example"}\NormalTok{),}
\NormalTok{  data,}
  \AttributeTok{overwrite =} \ConstantTok{TRUE}\NormalTok{,}
  \AttributeTok{field.types =} \FunctionTok{c}\NormalTok{(}
    \AttributeTok{col1 =} \StringTok{"varchar(50)"}\NormalTok{,}
    \AttributeTok{col2 =} \StringTok{"int"}\NormalTok{,}
    \AttributeTok{col3 =} \StringTok{"date"}
\NormalTok{  )}
\NormalTok{)}
\end{Highlighting}
\end{Shaded}

The dbWriteTable() function can be used to create tables in SQL Server from R.

To specify the naming convention of the table to be made, the below functions can be used:
- Database name: the database specified in the original sql\_connection using dbConnect()
- Schema and table name: the Id() function is used to specify the schema and table name

If the schema is not specified using the Id function, R will use the default schema for that database (usually dbo).

Use the overwrite argument to enable overwrites of the table.

Use the field.types argument to specify the data types to applied to each column when loading the data into R. If the field.types are not specified, SQL will estimate an appropriate data type to use - !warning! sql will use varchar(255) for all character columns, therefore it is recommended that data types are specified for larger tables.

\hypertarget{create-temporary-sql-table-from-r}{%
\section{Create temporary SQL table from R}\label{create-temporary-sql-table-from-r}}

\begin{Shaded}
\begin{Highlighting}[]
\FunctionTok{dbWriteTable}\NormalTok{(}
\NormalTok{  sql\_connection,}
  \StringTok{"\#\#BSOL\_XXXX\_example"}\NormalTok{,}
\NormalTok{  data}
\NormalTok{)}
\end{Highlighting}
\end{Shaded}

To create a temp table, the schema and database do not have to be specified.

\hypertarget{remove-sql-tables-from-r}{%
\section{Remove SQL tables from R}\label{remove-sql-tables-from-r}}

\begin{Shaded}
\begin{Highlighting}[]
\FunctionTok{dbRemoveTable}\NormalTok{(}
\NormalTok{  sql\_connection,}
  \FunctionTok{Id}\NormalTok{(}\AttributeTok{schema =} \StringTok{"dbo"}\NormalTok{, }\AttributeTok{table =} \StringTok{"BSOL\_XXXX\_example"}\NormalTok{)}
\NormalTok{)}
\FunctionTok{dbRemoveTable}\NormalTok{(sql\_connection, }\StringTok{"\#\#BSOL\_XXXX\_example"}\NormalTok{)}
\end{Highlighting}
\end{Shaded}

The dbRemoveTable() function can be used to remove tables from SQL server.

To remove static tables, the database needs to be specified via the sql connection and the schema must also be defined.

To remove temp tables, the database and schema do not need to be specified.

  \bibliography{book.bib,packages.bib}

\end{document}
